\documentclass{article}
\usepackage{amssymb, mathrsfs, enumitem}
\usepackage{tikz-cd}
\usepackage{multicol}
\usepackage{graphicx}
\usepackage{color}
\usepackage{amsthm}
\usepackage{hyperref}
\usepackage{verbatim}
\usepackage{caption}
\usepackage{subcaption}
\usepackage{amsmath, epsfig, colortbl}
\usepackage[breakable, skins]{tcolorbox}
\setlength{\parskip}{8pt}
\usepackage{geometry}
\geometry{
 a4paper,
 total={170mm,257mm},
 left=20mm,
 top=20mm,
 }
\begin{document}
\section{Important Terms}
\begin{itemize}

\item A {\it proposition} is a statement to which it is possible to assign a value of either true or false.
\item Let $p$ and $q$ be propositions. The {\it conjunction} of $p$ and $q$, denoted by $p\wedge q$ is a proposition that is true when both $p$ and $q$ are true, and false in any other condition.
\item Let $p$ and $q$ be propositions. The {\it disjunction} of $p$ and $q$, denoted by $p\vee q$, is a proposition that is true when at least one of $p$ or $q$ is true, and false if both $p$ and $q$ are false.
\item Let $p$ and $q$ be propositions. The {\it exclusive disjunction} of $p$ and $q$, denoted by $p\ \dot\vee\ q$ or $p\oplus q$, is a proposition that is true when exactly one of $p$ or $q$ is true, and false in any other condition.
\item Let $p$ be a proposition. The {\it negation} of $p$, denoted $\neg p$, is a proposition that is true when $p$ is false, and false when $p$ is true.
\item A {\it propositional formula} is a proposition constructed using propositional variables and logical operators.
\item But let's not be too hasty. I said above in {\it most} technical cases, but there are some writings that take $\wedge$ and $\vee$ to have {\it equal} precedence (like addition and subtraction in arithmetic). In that case, the meaning of our proposition would have been lost without parentheses. In general, I would recommend parenthesizing liberally here, so as not to confuse the reader as to which operations take priority in a given propositional formula.
\item A propositional formula is called a {\it tautology} if the formula is true, regardless of the truth values of the propositional variables from which it is constructed. Such a formula may be referred to as ``tautologically true.

\end{itemize}
\end{document}